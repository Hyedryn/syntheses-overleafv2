\documentclass[fr,license=none]{../../../eplsummary}

%\usepackage{MnSymbol}
\makeatletter
\newsavebox{\@brx}
\newcommand{\llangle}[1][]{\savebox{\@brx}{\(\m@th{#1\langle}\)}%
\mathopen{\copy\@brx\kern-0.5\wd\@brx\usebox{\@brx}}}
\newcommand{\rrangle}[1][]{\savebox{\@brx}{\(\m@th{#1\rangle}\)}%
\mathclose{\copy\@brx\kern-0.5\wd\@brx\usebox{\@brx}}}
\makeatother

\usepackage{bm}

\usepackage{pifont}
\newcommand{\cmark}{\ding{51}}%
\newcommand{\xmark}{\ding{55}}%

% widehat pour parent et hat quand = 0 sur Gamma_D
\newcommand{\hO}{\widehat{\Omega}}
\newcommand{\nh}{\widehat{n}}
\newcommand{\x}{\mathbf{x}}
\newcommand{\bxi}{\bm{\xi}}
\newcommand{\X}{\mathbf{X}}
\newcommand{\n}{\mathbf{n}}
\newcommand{\uh}{u^h}
\newcommand{\hu}{\hat{u}}
\newcommand{\huh}{\hat{u}^h}
\newcommand{\U}{\mathbf{\mathcal{U}}}
\newcommand{\Uh}{\mathbf{\mathcal{U}}^h}
\newcommand{\hU}{\hat{\mathbf{\mathcal{U}}}}
\newcommand{\hUh}{\hat{\mathbf{\mathcal{U}}}^h}

\newcommand{\inti}[2]{\langle#1,#2\rangle}
\newcommand{\intb}[2]{\llangle#1,#2\rrangle}
\newcommand{\intd}[2]{\llangle#1,#2\rrangle_D}
\newcommand{\intn}[2]{\llangle#1,#2\rrangle_N}

\hypertitle{Éléments finis}{4}{MECA}{1120}
{Benoît Legat\and Léa Paulus}
{Vincent Legat}

\section{Maillage non-structuré}
\subsection{Élément parent}
On divise le domaine compliqué $\Omega$ avec coordonnées $(x,y)$ en une multitude d'éléments plus simples,
e.g. triangles, quadrilatères.
On définit un élément parent $\hO$ dans lequel les coordonnées sont $(\xi,\eta)$, c'est l'élément de référence.
L'indice pour les éléments est typiquement $e$.
Le maillage est composé de $n$ \emph{sommets} et chaque élément de $\nh$ sommets.

$\Omega$ est l'union des éléments $\Omega^e$.
$\Omega^e$ peut s'exprimer comme un isomorphisme de l'élément parent $\hO$
\begin{align*}
  x(\xi,\eta) & = \sum_{i=1}^{\nh} X_i \phi_i(\xi,\eta)\\
  y(\xi,\eta) & = \sum_{i=1}^{\nh} Y_i \phi_i(\xi,\eta).
\end{align*}

Même si intuitivement, ça parait logique que les $\phi_i$ soient linéaires,
ce n'est pas possible pour des éléments quadrilatères.
Pour ces derniers on a des $\phi_i$ quadratique mais ils forment toujours une bijection si le quadrilatères $\Omega^e$ est convexe
(ce n'est pas un ``noeud papillon'').

\subsection{Interpolation: Valeurs nodales et fonction de formes locales}
On définit des noeuds dans le maillage et on interpolera une fonction $u$ à l'aide des valeurs qu'elle prend à ces noeuds.
Typiquement, on définit la position de noeuds dans l'élément parent et les noeuds de chaque élément du maillage sont obtenu par isomorphisme de ceux de l'élément parent.
Supposons ici pour simplifier qu'on définit un noeud à chaque sommet.

On approche une fonction $u$ par les \emph{valeurs nodales globales} $U_j = u(\X_j)$ que prend la fonction $u$ à chaque noeud
à l'aide de \emph{fonction de formes globales} $\tau_j$
\begin{equation}
  \label{eq:approx}
  u(\x) \approx u^h(\x) = \sum_{j=1}^n U_j\tau_j(\x).
\end{equation}
On aimerait au moins interpoler exactement aux valeurs nodales, c'est à dire $u(\X_j) = u^h(\X_j)$.
On impose donc que $\tau_j(\X_j) = 1$ et $\tau_j(\X_{j'}) = 0$ pour $j' \neq j$.

Tout comme avec les splines cubiques ou les Bsplines (qui sont les fonctions de bases des splines cubiques),
les fonctions de formes $\tau_j$ n'auront pas d'expression valable pour tout $\x \in \Omega$.
Elle seront définies différemment dans chaque élément.
\begin{itemize}
  \item Dans un élément $e$ dont aucun sommet n'est $j$,
    $\tau_j$ s'annule à tous les sommets.
    On y impose donc que $\tau_j$ est nul.
  \item Dans un élément $e$ dont un sommet est $j$,
    $\tau_j$ vaut 1 pour ce sommet et zéro pour les 2 autres.
    $\tau_j$ ne sera donc pas nul dans cet élément vaudra $\phi_i^e$ (voir plus loin)
    où $j$ est le sommet $i \in \{1,2,3\}$ de l'élément $e$.
\end{itemize}
On donnera typiquement au $\phi_i^e$ un expression polynômiale.
Le \emph{support} d'une fonction est l'ensemble des valeurs où elle est non-nulle.
Le support d'une fonction $\tau_j$ est l'union des éléments $\Omega^2$ où $j$ est un des sommets.
Les $\tau_j$ sont donc des fonctions polynomiales par morceau à support compact.

% TODO phi_i degré p

$u^h(\x)$ pour $\x \in \Omega^e$ est une interpolation des valeurs $U_i$ des sommets de l'éléments.
Tout comme avec l'interpolation avec les fonctions de Lagrange,
cette interpolation se fait à l'aide des \emph{fonctions de formes locales} $\phi_i^e$
\begin{align*}
  u^h(\x) & = \sum_{i=1}^{\nh} U_i \phi_i^e(\x)\\
          & = \sum_{i=1}^{\nh} U_i \phi_i(\bxi(\x))
\end{align*}
pour $i \in \Omega^e$.
Notons qu'on a $\x(\bxi)$ mais pas $\bxi(\x)$.
Heureusement, nous n'aurons jamais besoin de calculer la relation inverse $\bxi(\x)$ \cite[p.~10,13]{legat2015fem}.

Si en faisant varier les $U_i$, on sait avoir tous les polynômes de degré $p$ ou moins,
c'est à dire que les $\phi_i$ forment une famille génératrice de l'ensemble des polynômes de degré $p$ ou moins,
on dit que que les fonctions de formes sont complètes à l'ordre $p$.
Pour cela, il faut déjà que tous les monômes (e.g. $1, \xi, \eta, \xi^2, \xi\eta, \eta^2, \ldots$) de degré $p$ ou moins
se retrouvent dans au moins un $\phi_i$ (voir le triangle de Pascal pour les monômes en dimension 2).

Il est aussi adéquat de regarder si on sait former tous les polynômes d'ordre $p$ en $\x$.
C'est à dire, l'ensemble des polynômes en $\bxi$ qu'on sait obtenir en remplaçant la valeur ce $\x$ par $\x(\bxi)$.
On remarque que pour un élément quadrilatère, l'élément lagrangien à 9 noeuds et l'élément de Serendip à 8 noeuds sont tous les deux complets à l'ordre 2
mais seulement le premier est complets à l'ordre 2 en $\x$ \cite[p.~15]{legat2015fem}.

\begin{myrem}
  On distingue les sommets qui sont des éléments du maillage aux noeuds qui sont des points
  où sont définit une valeur nodale de l'approximation~\cite[p.~5,8]{legat2015fem}.

  De la même manière,
  il est important de ne pas confondre les fonctions utilisées
  pour la correspondance de $x,y$ avec $\xi,\eta$ et les fonctions de formes locales utilisées pour l'interpolation de $u$.

  Les premières dépendent uniquement du maillage et des sommets,
  elles servent juste à faire un isomorphisme entre les éléments et l'élément parent.
  Les deuxièmes dépendent des noeuds, elles servent à approximer $u$ en l'interpolant aux valeurs nodales,
  il y a un intérêt à augmenter le nombre de noeuds et donc le nombre de fonctions de formes locales et leur degré
  pour affiner l'interpolation ainsi que vérifier que de vérifier que les fonction de formes sont complètes.

  Elles sont souvent identiques lorsqu'on décide de mettre un noeud à chaque sommet ce n'est pas imposés.

  Elles doivent toutes les deux satisfaire la condition
  $\phi_i(\X_i) = 1$ et $\phi_i(\X_{i'}) = 0$ pour $i \neq i'$. mais pour l'une c'est à chaque sommet et pour l'autre à chaque noeud.

  En passant, on remarque que
  la condition $\sum_{i=1}^3 \phi_i(\x_i) = 1$ $\forall \x_i \in \Omega^e$ s'interprète différemment dans les deux cas.
  Pour les fonction utilisées pour l'isomorphisme, il faut l'imposer car elle garanti que
  $\x$ est une combinaison convexe des $\X_i$.
  C'est important car $\Omega^e$ est l'enveloppe convexe des sommets $X_i$.
  Si on avait $\sum_{i=1}^3 \phi_i(\x_i) \neq 1$,
  certains $\x$ pourraient sortir du triangle.
  Pour les fonction de formes locales par contre, la condition $\sum_{i=1}^3 \phi_i(\x_i) = 1$
  assure juste qu'on interpole parfaitement les fonctions $u$ constantes.
\end{myrem}

\subsection{Intégration}
On peut estimer l'intégrale d'une fonction dans un élément $e$ à l'aide de la somme des valeurs
que prend la fonctions à différents points pondérés par des poids.
Supposons qu'on utilise $p$ points.
On doit choisir où se situent ces $p$ points ainsi que le poids.
En $N$ dimension, ça fait $(N+1)p$ degrés de libertés.

Dans la méthode de Gauss-Legendre, on impose un degré de précision $d$.
C'est à dire qu'on veut intégrer exactement tous les polynômes de degré $d$ ou moins.
Comme l'intégrale et notre approximation sont linéaires en la fonction à intégrer,
on peut tout aussi bien imposer qu'on intègre parfaitement tous les monômes,
i.e. $1, \xi, \eta, \zeta, \xi^2, \xi\eta, \ldots$.
En effet, les monômes forment la base canonique de l'espaces des polynômes.

On dit qu'un monôme est de degré $d$ si la somme des exposants des différentes variables est $d$.
Par exemple, $\xi^3\eta^2\zeta^3$ est de degré $3+2+3 = 8$.
Un polynôme de degré $d$ est une combinaison linéaire de monômes de degré $d$ ou moins avec au moins un monôme de degré $d$.
La dimension de l'ensemble des polynômes de degré $d$ est égal au nombre de monômes différent.
Il est d'ailleurs assez cocasse de remarquer que le nombre de monômes de degré $d$ ou moins à $N$ variables est \emph{égal}
au nombre de monômes de degré $d$ à $N+1$ variables (on obtient une bijection en ajoutant une variable avec comme exposant ce qu'il manque pour faire $d$).
En se souvenant du cours FSAB1101, on voit directement que le nombre de monômes de degré $d$ à $N+1$ variables est $B^*(N+1,d) = {N+d \choose d}$.
Pour avoir un degré de précision $d$ en dimension $N$, il faut donc intégrer parfaitement ${N+d \choose d} = {N+d \choose N}$ monômes.

Regardons ce que ça donne en dimension 2.
On a $3p$ degré de liberté, et ${2+d \choose 2} = (d+1)(d+2)/2 = 1+2+3+\cdots+d+(d+1)$ contraintes.
Pour avoir un degré de précision $d$ avec Gauss-Legendre il faut donc choisir un nombre de points $p$ tel que (voir \tabref{gl2})
\[ \frac{(d+1)(d+2)}{2} \leq 3p. \]
\begin{table}
  \centering
  \begin{tabular}{cccc}
    $d$ & $(d+1)(d+2)/2$ & $p$ minimum & $3p \stackrel{?}{=} (d+1)(d+2)/2$\\
    1  & 1  & 1\\
    2  & 3  & 1 & \cmark\\
    3  & 6  & 2 & \cmark\\
    4  & 10 & 4\\
    5  & 15 & 5 & \cmark\\
    6  & 21 & 7 & \cmark\\
    7  & 28 & 10\\
    8  & 36 & 12 & \cmark\\
    9  & 45 & 15 & \cmark\\
    10 & 55 & 19\\
  \end{tabular}
  \caption{Nombre minimum de points pour Gauss-Legendre en 2 dimensions.}
  \label{tab:gl2}
\end{table}

On définit une règle d'intégration en donnant les poids et les coordonnées $\xi, \eta$ des $p$ points dans l'élément parent.
Pour intégrer dans un élément $e$, on se ramène à l'élément parent à l'aide de
\[ \int_{\Omega^e} f(x,y) \dif x \dif y = \int_{\hO} f(x(\xi,\eta), y(\xi, \eta)) |J_e| \dif \xi \dif \eta. \]
où $J_e$ est donné par
\[
  J_e = \det ...
\] % TODO

\begin{myrem}
  degré de précision est sur $fJ_e$, pas sur $f$ ! % TODO
\end{myrem}


\section{Éléments finis pour des problèmes elliptiques}
\subsection{Différentes formulations}
Soit $\U_s, \U, \hU$ trois espaces de fonctions où
les fonctions de $\U_s$ sont deux fois différentiables,
les fonctions $v \in  \U$ sont telles que $v(\x) = t(\x)$ $\forall x \in \Gamma_D$ et
les fonctions $v \in \hU$ sont telles que $v(\x) = 0$ $\forall x \in \Gamma_D$.

On pourrait choisir que $\U = \{v \in \U_s | v(\x) = t(\x), \forall \x \in \Gamma_D\}$ et $\hU = \{v \in \U_s | v(\x) = 0, \forall \x \in \Gamma_D\}$ mais on ajoutera souvent d'autres restrictions~\cite[p.~22]{legat2015fem}.

\begin{myprob}[Strong formulation of Elliptic problem]
  \label{prob:strong}
  Trouver $u(\x) \in \U_s$ tel que
  \begin{align*}
    \grad \cdot (a \grad u) + f & = 0, & \forall \x \in \Omega,\\
    \n \cdot (a \grad u) & = g,        & \forall \x \in \Gamma_N,\\
    u & = t,                           & \forall \x \in \Gamma_D
  \end{align*}
  où $a$ est strictement positif~\cite[pp.~24--25]{legat2015fem}.
\end{myprob}

On introduit les 4 produits scalaires suivants%
\footnote{Le $\cdot$ de la première définition est un produit scalaire pour si jamais $f$ et $g$ sont des vecteur (car on a pris un gradient par exemple comme avec $a$ définit juste après)}
\begin{align*}
  \inti{f}{g} & = \int_\Omega f \cdot g \dif \Omega\\
  \intb{f}{g} & = \int_{\partial\Omega} f g \dif s\\
  \intn{f}{g} & = \int_{\Gamma_N} f g \dif s\\
  \intd{f}{g} & = \int_{\Gamma_D} f g \dif s
\end{align*}
% TODO essentielle, naturelle
% TODO matrice de raideur, forces nodales

et les formes suivantes (ce sont des fonctions de fonction!)
\begin{align*}
  a(v,w) & \eqdef \inti{\grad v}{a \grad w}\\
  b(v) & \eqdef \inti{f}{v} + \intn{g}{v}\\
  J(v) & \eqdef \frac{1}{2} a(v,v) - b(v)
\end{align*}
pour $v,w \in \U$.

\begin{myprob}[Weak formulation of Elliptic problem]
  \label{prob:weak}
  Trouver $u(\x) \in \U$ tel que
  \begin{align*}
    a(\hu, u) & = b(\hu), & \forall \hu \in \hU,\\
  \end{align*}
\end{myprob}

\begin{myprob}[Minimization problem of Elliptic problem]
  \label{prob:min}
  Trouver $u(\x) \in \U$ tel que
  \begin{align*}
    J(u) & = \min_{v \in \U} J(v)
  \end{align*}
  c'est à dire trouver une fonction $u \in \U$ qui minimise $J$.
\end{myprob}

Problème~\ref{prob:weak} est une formulation faible de Problème~\ref{prob:strong}.
C'est à dire que toute solution de Problème~\ref{prob:strong} est également solution de Problème~\ref{prob:weak}.
Problème~\ref{prob:weak} et Problème~\ref{prob:min} sont des formulation équivalentes si $a$ est symétrique, ce qui est le cas pour un problème élliptique, Problème~\ref{prob:min} ne pourra pas être utilisé lorsqu'il y aura un terme $\grad u$ dans l'EDP.

\subsection{Approximation}
En plus de travailler avec une formulation faible du problème de départ,
on va travailler avec un modèle approximé à l'aide de l'approximation du $u$ donnée en \eqref{eq:approx}.
Le Problème~\ref{prob:weak} s'appelle alors méthode de Galerkin et le Problème~\ref{prob:min} la méthode de Ritz.

Dans nos deux formulations, on remplace $\U$ par $\Uh$ et $\hU$ par $\hUh$

Soit $I$ l'ensemble des noeuds qui ne sont pas à la frontière où on impose une condition de Dirichlet, i.e. $\X_i \notin \Gamma_D$.
On remarque avec \eqref{eq:approx} que $\Uh$ est engendré par la base $\tau_1, \ldots, \tau_n$
mais pour $v \in \hUh$, $V_i = 0$ pour $i \in \Gamma_D$.
Dans la base de $\hUh$, il n'y a donc que les $\tau_i$ pour $i \in I$.

La condition du Problème~\ref{prob:weak} devient
\begin{align*}
  a(\huh, \uh) & = b(\huh), & \forall \huh \in \hUh\\
  \inti{\grad \huh}{a \grad \uh} & = \inti{f}{\huh} + \intn{g}{\huh}, & \forall \huh \in \hUh\\
  \inti{\grad \tau_i}{a \sum_{j=1}^n U_j \tau_j} & = \inti{f}{\tau_i} + \intn{g}{\tau_i}, & \forall i \in I\\
  \sum_{j = 1}^n \inti{\grad \tau_i}{a \grad \tau_j} U_j & = \inti{f}{\tau_i} + \intn{g}{\tau_i}, & \forall i \in I\\
  \sum_{j = 1}^n a(\grad \tau_i, \grad \tau_j) U_j & = b(\tau_i), & \forall i \in I.
\end{align*}
Pour passer de la 2ième à la 3ième ligne, on a utilisé la linéarité de l'équation en $\huh$ pour dire que la condition est satisfaite
pour tout $\huh$ si elle est satisfaite pour tout élément de la base de $\hUh$.
Si on ajoute les conditions $U_j = t(\X_j)$ pour les noeuds $j \notin I$, on obtient un système linéaire de $n$ équations et $n$ inconnues $U_j$.

Soit $U$ le vecteur des inconnues et $A$ et $b$ tels que le système linéaire est $AU = b$.

On voit que $A_{ij} = a(\grad\tau_i, \grad \tau_j)$ si $i \in I$ et $A_{ij} = \delta_{ij}$ sinon.
Notons que si $i$ et $j$ ne sont pas les noeuds d'un même élément, le support de $\tau_i$ et $\tau_j$ n'a aucune intersection et $a(\tau_i, \tau_j) = 0$.
Sinon on utilise
\[ a(\grad\tau_i, \grad\tau_j) = \sum_e a(\grad \phi_{i'}^e, \grad \phi_{j'}^e)  \]
où on somme sur tous les éléments $e$ qui ont le noeuds $i$ et le noeuds $j$.
On leur donne respectivement les indices locaux $i'$ et $j'$.

De la même manière, $b_i = b(\tau_i)$ si $i \in I$ et $b_i = t(\X_i)$ sinon.

En faisant le résonnement pour le Problème~\ref{prob:min}, on obtient le même système à résoudre.
On peut aussi obtenir ce résultat avec une méthode de résidus pondérés~\cite[pp.~27--29]{legat2015fem}.

\subsection{Analyse}
On remarque que $a$ est une forme bilinéaire et symétrique (ce n'aurait pas été le cas s'il y avait eu un terme en $\grad u$ dans l'EDP).
Pour que $a$ soit un produit scalaire, il faut également que $a(\grad v, \grad v) \Rightarrow v = 0$ pour $v \in \U$.

On remarque que $a(\grad v, \grad v) = 0 \Rightarrow \int_\Omega a(\grad v)^2 = 0$.
Si $a$ est strictement positif et que les fonction de $\U$ sont continues, ça signifie que $\grad v = 0$.
Seulement, cela n'implique pas que $v = 0$, en effet, souvenons-nous qu'une primitive est définie à une constante prêt.
Par contre si on impose que pour qu'une fonction $u \in \U$, il faut que $u = 0$ pour tout $\x \in \partial\Omega$,
on a bien que $v$ doit être la fonction nulle.

$a$ est donc un produit scalaire si $\U = \{v \in H^1(\Omega) \mathbin{\mathrm{et}} v(\x) = 0, \forall \x \in \partial \Omega\}$.
En fait, pour un tel $\U$, $a$ est également coercive.
Comme $a$ est continue et que $b$ est linéaire et continue,
le Théorème de Lax-Milgram permet de garantir que la formulation faible (Problème~\ref{prob:weak} ou Problème~\ref{prob:min}) ont une solution unique.

\section{Éléments finis pour des problèmes d'advection-diffusion}
\section{Théorie de la meilleure approximation}

\biblio

\end{document}
